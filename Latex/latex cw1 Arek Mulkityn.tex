\documentclass[12pt, letterpaper, titlepage]{article}
\usepackage[left=3.5cm, right=2.5cm, top=2.5cm, bottom=2.5cm]{geometry}
\usepackage[MeX]{polski}
\usepackage[utf8]{inputenc}
\usepackage{graphicx}
\usepackage{enumerate}
\usepackage{amsmath} %pakiet matematyczny
\usepackage{amssymb} %pakiet dodatkowych symboli
\title{Pierwszy dokument LaTeX}
\author{Arek Mulkityn}
\date{Październik 2022}
\begin{document}
\maketitle

\newpage

\begin{enumerate}
\item pierwszy
\item drugi
 
\end{enumerate}

\begin{enumerate}[A]
\item jeden
\item dwa
\end{enumerate}

\paragraph{Paragraph Nr.1}

\section{\textbf{ABSTRACT}

\textit{The use of airborne laser scanning for inventory and monitoring of landslides in the Łaśnica area (Lanckorona Commune),Wielickie Foothills, Outer Carpathians.
\underline{A b s t r a c t}. In recent years a regular activity has been taken for the registration and monitoring of areas at risk of mass movements and landslides throughout Poland. Extensive inventory work in the sites predisposed to occurrence of landslides, initiated a search in order to improve traditional methods of mapping landslides. The traditional method relies mainly on the analysis of topographic maps, geological and geomorphological mapping in the field. For areas of extreme danger the newer mainly non-invasive methods were tried to be used such as a satellite or aerial photos. In this article have been also tested one of the more modern methods of three dimensional imaging earth – Airborne Laser Scanning. This method is applicable to the selected landslide in the region of Łaśnica (Municipality Lanckorona). Amajor advantage of the method is the ability to filter out vegetation and other objects on the ground, which results in precise terrain model. Multiple imaging using laser scanning method, allows to obtain a precise differential model, thus in effect information on landslide activity.}
}



\subsection{\textbf{ABSTRACT}

The use of airborne laser scanning for inventory and monitoring of landslides in the Łaśnica area (Lanckorona Commune),Wielickie Foothills, Outer Carpathians.
A b s t r a c t. In recent years a regular activity has been taken for the registration and monitoring of areas at risk of mass movements and landslides throughout Poland. Extensive inventory work in the sites predisposed to occurrence of landslides, initiated a search in order to improve traditional methods of mapping landslides. The traditional method relies mainly on the analysis of topographic maps, geological and geomorphological mapping in the field. For areas of extreme danger the newer mainly non-invasive methods were tried to be used such as a satellite or aerial photos. In this article have been also tested one of the more modern methods of three dimensional imaging earth – Airborne Laser Scanning. This method is applicable to the selected landslide in the region of Łaśnica (Municipality Lanckorona). Amajor advantage of the method is the ability to filter out vegetation and other objects on the ground, which results in precise terrain model. Multiple imaging using laser scanning method, allows to obtain a precise differential model, thus in effect information on landslide activity.
}

\subsubsection{\textbf{ABSTRACT}

The use of airborne laser scanning for inventory and monitoring of landslides in the Łaśnica area (Lanckorona Commune),Wielickie Foothills, Outer Carpathians.
A b s t r a c t. In recent years a regular activity has been taken for the registration and monitoring of areas at risk of mass movements and landslides throughout Poland. Extensive inventory work in the sites predisposed to occurrence of landslides, initiated a search in order to improve traditional methods of mapping landslides. The traditional method relies mainly on the analysis of topographic maps, geological and geomorphological mapping in the field. For areas of extreme danger the newer mainly non-invasive methods were tried to be used such as a satellite or aerial photos. In this article have been also tested one of the more modern methods of three dimensional imaging earth – Airborne Laser Scanning. This method is applicable to the selected landslide in the region of Łaśnica (Municipality Lanckorona). Amajor advantage of the method is the ability to filter out vegetation and other objects on the ground, which results in precise terrain model. Multiple imaging using laser scanning method, allows to obtain a precise differential model, thus in effect information on landslide activity.
}




\end{document}